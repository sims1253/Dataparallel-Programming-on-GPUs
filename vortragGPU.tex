\documentclass{beamer}
\usepackage{beamerthemelined} 
\usepackage{amsmath}
\usepackage{mathtools}
\usepackage{listings}
\usepackage{graphicx}
\usepackage{epstopdf}

\setbeamertemplate{footline}[frame number]

\lstdefinestyle{customc}{
  belowcaptionskip=1\baselineskip,
  breaklines=true,
  frame=L,
  xleftmargin=\parindent,
  language=C,
  showstringspaces=false,
  basicstyle=\footnotesize\ttfamily,
  keywordstyle=\bfseries\color{green!40!black},
  commentstyle=\itshape\color{purple!40!black},
  identifierstyle=\color{blue},
  stringstyle=\color{orange},
}

%\usepackage{ stmaryrd }

%\usetheme{Boadilla} % Pretty neat, soft color.
%\usetheme{default}
%\usetheme{Warsaw}
%\usetheme{Bergen} % This template has nagivation on the left
%\usetheme{Frankfurt} % Similar to the default 
%with an extra region at the top.
%\usecolortheme{seahorse} % Simple and clean template
%\usetheme{Darmstadt} % not so good
% Uncomment the following line if you want %
% page numbers and using Warsaw theme%
 %\setbeamertemplate{footline}[page number]
%\setbeamercovered{transparent}
\setbeamercovered{invisible}
% To remove the navigation symbols from 
% the bottom of slides%
\setbeamertemplate{navigation symbols}{} 
%

\usepackage{graphicx}
%\usepackage{bm}         % For typesetting bold math (not \mathbold)
%\logo{\includegraphics[height=0.6cm]{yourlogo.eps}}
%
\title[Dataparallel Programming on GPUs]{Dataparallel Programming on GPUs}
\author{Daniel Rembold, Maximillian Scholz}
\institute[TUHH]
{
Technische Universit{\"a}t Hamburg Harburg \\
\medskip
{\emph{daniel.rembold@tuhh.de , maximillian.scholz@tuhh.de}}
}
\date{\today}
% \today will show current date. 
% Alternatively, you can specify a date.
%


\begin{document}
\renewcommand{\figurename}{Abbildung}
%
\begin{frame}
\titlepage
\end{frame}
%

\begin{frame}
\frametitle{Inhaltsverzeichnis}

\begin{enumerate}
		\item Aufgabenstellung \\

	      	\item Einfache Implentierung \\
	      	
	      	\item NVIDIA-Template   \\ 
	      		
	      	\item Optimierung  \\ 
	  
	      	\item Laufzeit- und Performancemessung \\

		\item Fazit
\end{enumerate}

\end{frame}

%%%%%%%%%%%%%%%%%%%%%%%%%%%%%%%%%% BEGIN INTRODUCTION %%%%%%%%%%%%%%%%%%%%%%%%%%%%%%%%%
\section{Aufgabenstellung}

\begin{frame}
\frametitle{Aufgabenstellung}

M\"oglichst effiziente Implementierung einer Matrixmultiplikation in row-major und column-major Format .\newline

Quadratische Matrizen der Dimension $32 * k , k \in \{ 1,2,4,\dots,128 \} $ 



\end{frame}

%%%%%%%%%%%%%%%%%%%%%%%%%%%%%%%%%%%%%%%%%%%%%%%%%%%%

\section{Einfache Implentierung}



%%%%%%%%%%%%%%%%%%%%%%%%%%%%
\begin{frame}[fragile]
\frametitle{Naiver Algoritmus}

C-Code f\"ur die CPU .. keine Parallelit\"at

\begin{lstlisting}[style=customc,caption=Matrixmultiplication in C]
int i,j,k;

for( i = 0, i<Ndim; i++){
  for( j = 0, j<Mdim;j++){
    for( k = 0; k<Pdim; k++){
      C[i*Ndim+j] += A[i*Ndim+k] * B[k*Pdim+j]; 
    }
  }
}
\end{lstlisting}



\end{frame}


% % % % % % % % % % % % % % % % % % % % % % % % % % % % %

\begin{frame}[fragile]
\frametitle{Einfache Implementierung}
 
Einfache Implementierung in OpenCL
\begin{lstlisting}[style=customc,caption=Simplecode]
int i,j,k;
i = global_id(0);
j = global_id(1);
  for(k=0; k<Pdim; k++){
    C[i*Ndim+j] += A[i*Ndim+k] * B[k*Pdim+j]; 
  }
\end{lstlisting}


%TODO Laufzeiten hier hochladen!

\end{frame}
%%%%%%%%%%%%%%%%%%%%%%%%%%%

% % % % % % % % % %  % %  % % %  % % % %  % %%  % % %  

\section{NVIDIA-Template}
\begin{frame}
\frametitle{NVIDIA-Template}

Zerlegung der Matrix in Bl\"ocken \\

Verwendung von lokalem Speicher \\

%TODO Code snippet hochladen

%TODO Laufzeitmessung hochladen abh. von Blockgröße
 

\end{frame}



\section{Optimierung}
\begin{frame}
\frametitle{Supermulti}

Weniger Threads, aber mehr Arbeit pro Thread -> Mehr Output pro Thread \\

Werte werden in Threads wiederverwendet

%TODO: Codesnippet hochladen 

\end{frame}

\section{Laufzeit- und Performancemessung}
\begin{frame}
\frametitle{Laufzeitmessung auf ATI und NVIDIA}
 %TODO MATLAB hochladen
\end{frame}

\begin{frame}
\frametitle{Performancemessung auf ATI und NVIDIA}
 %TODO MATLAB hochladen
	%\begin{figure}[c]
	%\includegraphics[width=12cm]{./Bilder/performance.eps}
	%\caption{PERFORMANZE}
	%\end{figure}
\end{frame}


\section{Fazit}
\begin{frame}
\frametitle{Begründung der Ergebnisse und m\"ogl. Verbesserungen}

 
\end{frame}













%%%%%%%%%%%%%%%%%%%%%%%%%%%%%%%%%
%%%%%%%%%%%%%%%%%%%%%%%%%%%%%%%%%
%%%%%%%%%%%%%%%%%%%%%%%%%%%%%%%%%

\begin{frame}
\frametitle{Quellen}
%TODO Quellen vollmachen
\footnotesize{
\begin{thebibliography}{99}
 \bibitem[Label1, 2010]{key1} Julien Arino, Jonathan R.Davis, David Hartley, Richard Jordan (2005)
 \newblock A multi-species epidemic model with spatial dynamics.
 \newblock \emph{Mathematical Medicine and Biology} 22, 129 -- 142.
\end{thebibliography}
}


\footnotesize{
\begin{thebibliography}{99}
 \bibitem[Label1, 2010]{key1} Jan W. Pr{\"u}ss,
 Roland Schnaubelt, Rico Zacher (2008)
 \newblock Mathematische Modelle
 in der Biologie - Deterministische homogene Systeme.
 \newblock ISBN 978-3-7643-8436-4
\end{thebibliography}
}


\end{frame}
 
 
 
\begin{frame}
\centerline{Vielen Dank f{\"u}r eure Aufmerksamkeit!}
\end{frame}
% End of slides
\end{document}