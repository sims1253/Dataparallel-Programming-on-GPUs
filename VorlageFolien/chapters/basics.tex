\section{Basic Work}


\subsection{Some of the basic work}

\begin{frame}
\frametitle{Great Frame}
A multidimensional array is a data structure which holds a set of data all of the same type whose elements are arranged in a rectangular pattern\footcite{ISO:2010:Fortran}.
\begin{Definition}[$N$-Way Array]
\label{def:array}
An $n$-way array is really nice!
\end{Definition}
\end{frame}


\begin{frame}
\frametitle{Notation}
Arrangement of the elements of a multidimensional array $\mathbf{A}$ of rank $n$ is given by
\begin{equation}\label{equ:array}
\mathbf{A} = (a_{x_1,\dots,x_n}).
\end{equation}%
Now this is complicated!
\begin{example}[Arrangement]
\label{ex:array}
I like arrays!
\begin{align*}
\mathbf{A} &=
\begin{pmatrix}
a_{0,0,0} & a_{0,1,0} & & a_{0,0,1} & a_{0,1,1} & & a_{0,0,2} & a_{0,1,2} \\
a_{1,0,0} & a_{1,1,0} & & a_{1,0,1} & a_{1,1,1} & & a_{1,0,2} & a_{1,1,2} \\
a_{2,0,0} & a_{2,1,0} & & a_{2,0,1} & a_{2,1,1} & & a_{2,0,2} & a_{2,1,2} \\
a_{3,0,0} & a_{3,1,0} & & a_{3,0,1} & a_{3,1,1} & & a_{3,0,2} & a_{3,1,2}
\end{pmatrix}.
\end{align*}
\end{example}
\end{frame}


\subsection{Other basic work}


